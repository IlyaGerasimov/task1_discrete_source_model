\documentclass{article}
\usepackage{graphicx}
\usepackage[russian]{babel}
\usepackage[T1,T2A]{fontenc}
\usepackage[utf8]{inputenc}
\usepackage{amsmath, amsfonts}

\title{Задание 1}
\author{И. Герасимов}
\date{}
\begin{document}
\maketitle

\section{Использование программы}

\begin{verbatim}
python3.8 main.py -f FILE [-n N] [-s SEQUENCE]
\end{verbatim}

Наличие параметра SEQUENCE определяет выбранный режим. Если его нет, то первый, если есть, то второй. 

\section{Обоснование интерпретации выходных данных}
	Моделирование источника происходит следующим образом:
	\begin{enumerate}
		\item Случайно выбираются два числа от 0 до 1.
		Одно определяет выбранную модель согласно распределению переключателя.
		Второе определяет выбранный символ согласно распределению выбранной модели.
		\item Определение происходит следующим образом: программа поочередно проходит по всем моделям/символам и суммирует вероятность.
		Если на очередной итерации полученная сумма превысила случайно выбранное число, то берется модель/символ этой итерации.
	\end{enumerate}
	
	\paragraph{пример.} Пусть источник находится на переключателе с распределением 0.7 для монеты-1 и 0.3 для монеты-2.
	Было выбрано число 0.72453425. На первой итерации сумма равна 0.7 и она меньше выбранного числа. На второй итерации сумма стала больше и выбрана монета-2.
	
	Теперь нужно определить символ по монете-2 с распределением 0.6 и 0.4. Было выбрано число 0.21353464 и на первой итерации получили большую сумму. Выбран первый символ.
	
	Практическое (по полученной выборке длины $N$ от источника) вычисление вероятности какого-либо сообщения длины $l$ выполняется подсчетом количества появлений этого сообщения на общее количество сообщений в выборке длины $N$ (то есть всего $N - l + 1$)
	
	\section{Пример стационарного источника}
	
	Описание источника можно найти в файле station.json.
	Имеем два переключателя и две монеты.
	В нечетный момент времени используется первый переключатель.
	В четный момент времени используется второй переключатель.
	Распишем вероятности каждого из символов в нечетный и четный моменты. Поскольку требуется построение стационарного источника, мы хотим, чтобы эти вероятности совпали (то есть нет зависимости от времени).
	
	Пусть для первого переключателя вероятности моделей равны, а для второго - 0.7 и 0.3 соответственно.
	Пусть:
	\begin{description}
		\item[$p$] вероятность первого символа в первой модели. $1-p$ - вероятность второго символа.
		\item[$q$] вероятность первого символа во второй модели. $1-q$ - вероятность второго символа.
	\end{description}
	
	Получаем следующие уравнения для стационарного источника:
	\[0.5p + 0.5q=0.7p + 0.3q\]
	\[0.5(1-p)+0.5(1-q)=0.7(1-p)+0.3(1-q)\]
	Решением будет $p=q$. То есть, если у нас в разные момент времени разные переключатели, то модели должны быть одинаковыми. В файле указан пример для $p=q=0.4$.
	
	То есть, если мы в зависимости от времени выбираем две разных монеты, мы все равно можем получить независимость вероятностей от времени.
	
	Отметим, что источник также является источником без памяти и, соответственно, эргодическим.
	
	\section{Пример нестационарного источника}
	
	Пусть имеется станок, который выпускает некоторую деталь. Деталь, созданная станком может оказаться дефектной. С каждой выпущенной деталью, качество станка уменьшается. Поэтому через некоторое число деталей (например после 4 итераций) его заменяют на новый.
	
	Получаем следующий нестационарный источник, описанный в файле nonstation.json.
	Пусть символ <<good>> означает, что деталь в хорошем состояний, а <<bad>> - деталь бракованная.
	Опишем 4 переключателя, при переходе по которым вероятность выпуска хорошей детали уменьшается, а бракованной увеличивается. Далее источник переходит к первому переключателю, что означает замену станка.
	
	\section{Пример неэргодического источника}
	
	Пусть мы хотим реализовать источник, который будет выдавать ответы <<да>>, <<нет>>, <<не знаю>> для дистанционного тестирования в зависимости от настроения сдающего. Настроение сдающего будет выбираться произвольно из множества <<оптимист>>, <<пессимист>>, <<студент>>. Если настроение <<оптимист>>, то предпочтение отдается ответу <<да>>. <<Пессимист>> - предпочтение ответу <<нет>>. <<студент>> - предпочтение ответу <<не знаю>>.
	
	Источник можно описывать следующим образом (файл nonergodic.json).
	имеется единственный переключатель, реализующий равновероятный выбор одной из трех моделей. В первой и второй моделях символы <<да>> и <<нет>> соответственно имеют большую вероятность, чем другие. В третьей всегда выбирается символ <<не знаю>>.
	
	Рассмотрим верятности появления некоторого ответа:
	
	$Pr\left[\text{<<да>>}\right] = \frac{1}{3}0.8+\frac{1}{3}0.1 = 0.9\frac{1}{3}$
	
	$Pr\left[\text{<<нет>>}\right] = \frac{1}{3}0.1+\frac{1}{3}0.8 = 0.9\frac{1}{3}$
	
	$Pr\left[\text{<<не знаю>>}\right] = \frac{1}{3}0.1+\frac{1}{3}0.1 + \frac{1}{3} = 1.2\frac{1}{3}$
	
	Что не совпадает с частотами по каждой реализации.
	
	\section{Предлагаемое развитие предложенного формата для описания источника с памятью}
	
	Предлагается следующее (реализуется в задании 2):
	
	В source указывать не строку, а словарь, например:
	\begin{verbatim}
{
  ``switch'': switch_1,
  ``input'': [
    {
    	``code'': [`0'],
    	``next'': 0
    }
  ]
}
	\end{verbatim}

\begin{itemize}
\item По ключю switch указывается имя переключателя;
\item input - список возможных значений памяти. Состоит из словарей;
\item По ключю code указывается предыдущее требуемое сообщение - чему должна быть равна память.
\item По ключю next указывается номер элемента в списке source, который должен обрабатываться, если память равна содержимому по ключю code.
\end{itemize}

Для источников без памяти input равен пустому списку.


\end{document}	